\documentclass[11pt,a4paper]{article}

\usepackage[utf8]{inputenc}
\usepackage[T1]{fontenc}
\usepackage{amsmath,amssymb,amsthm}
\usepackage{booktabs}
\usepackage{graphicx}
\usepackage[margin=1in]{geometry}
\usepackage{hyperref}
\usepackage{xcolor}
\usepackage{fancyhdr}
\usepackage{float}
\usepackage{enumitem}

\pagestyle{fancy}
\fancyhf{}
\fancyhead[C]{\small\itshape Satellite Primes: Prime Gap Structure near $Q(n) = n^{47} - (n{-}1)^{47}$}
\fancyfoot[C]{\thepage}
\renewcommand{\headrulewidth}{0.4pt}

\newtheorem{theorem}{Theorem}[section]
\newtheorem{proposition}[theorem]{Proposition}
\newtheorem*{remark}{Remark}

\hypersetup{
    colorlinks=true,
    linkcolor=blue!60!black,
    citecolor=blue!60!black,
    urlcolor=blue!70!black,
}

\title{\textbf{Satellite Primes: The Local Prime Landscape} \\[6pt]
\textbf{around Giant Primes from} $\boldsymbol{Q(n) = n^{47} - (n-1)^{47}}$ \\[6pt]
\large A Cram\'er-Model Validation at 500-Digit Scales}

\author{Ruqing Chen \\[4pt]
\textit{GUT Geoservice Inc., Montr\'eal, Canada} \\
\texttt{ruqing@hotmail.com}}

\date{February 2026}

\begin{document}

\maketitle
\thispagestyle{fancy}

\begin{center}
\textit{Subject: Computational Number Theory / Prime Gap Statistics} \\[4pt]
\textit{Part IV of the Titan Project}
\end{center}

\begin{abstract}
We report the first large-scale empirical study of the local prime
landscape surrounding giant primes $P = Q(n) = n^{47} - (n{-}1)^{47}$,
probing the nearest primes within a radius $R = 5000$ of each~$P$.
For 2{,}107 main-star primes~$P$ of 494--521 digits
($n \in [5.3 \times 10^{10},\, 2.0 \times 10^{11}]$), a total of
\textbf{9{,}012 satellite primes} $P - k$ were discovered via
probabilistic primality testing.

The satellite count per star follows a Poisson distribution with
$\lambda = 4.28$ and dispersion index 1.07---a textbook-quality
fit across all 12 bins including the ${\sim}28$ zero-satellite
stars recovered from a data-aggregation artifact.
The gap distribution within $[2, 5000]$ is uniform
($\chi^2$ test, $p = 0.31$), confirming the Cram\'er random model at
500-digit scales.  The nearest-satellite CDF matches the Cram\'er
exponential $1 - \exp(-k / 3\ln P)$ to within 1--3\% across the
full range.

All gaps~$k$ satisfy $k \equiv 0$ or $2 \pmod{6}$, a consequence
of the fixed residues $Q(n) \equiv 1 \pmod{2}$ and
$Q(n) \equiv 1 \pmod{3}$.  This creates a \emph{forbidden residue
lattice} that eliminates $2/3$ of even gaps, including all
$k \equiv 4 \pmod{6}$.

A confirmation scan of all 2{,}992 main-star primes at
radius $R = 100$ reveals \textbf{7 twin prime pairs} ($k = 2$)
and \textbf{7 sexy prime pairs} ($k = 6$)---500-digit primes
separated by only 2 or 6.  Both counts agree with the conditional
Hardy--Littlewood expectation of $E \approx 7.2$ to within
$0.1\sigma$, providing a precision validation of the Bayesian
concentration principle: the fixed residue
$P \equiv 1 \pmod{3}$ doubles the conditional rate for
$k \equiv 2 \pmod{6}$ gaps, exactly compensating the smaller
unconditional singular series, so that
$\mathcal{S}_{\text{cond}}(k{=}2) = \mathcal{S}_{\text{cond}}(k{=}6)
= 2.64$.  The equality
$N_{\text{twin}} = N_{\text{sexy}} = 7$ is a direct empirical
confirmation of this identity.

These results demonstrate that the local prime environment near
algebraically structured giant primes is statistically indistinguishable
from the Cram\'er random model, ruling out any detectable ``repulsion''
or ``attraction'' effects from the polynomial origin of these primes.

\medskip
\noindent\textbf{Keywords:}
satellite primes, prime gaps, Cram\'er model, Poisson statistics,
giant primes, forbidden residue lattice, twin primes, sexy primes,
conditional Hardy--Littlewood, computational number theory

\medskip
\noindent\textbf{MSC 2020:} 11N05, 11A41, 11Y11, 11Y16
\end{abstract}

\newpage

%==========================================================================
\section{Introduction}
%==========================================================================

The distribution of prime gaps is among the deepest questions in
analytic number theory.  Cram\'er's probabilistic model~\cite{Cramer1936}
posits that primes near~$x$ behave like independent random events of
density $1/\ln x$, predicting that prime gaps near~$x$ follow an
exponential distribution with mean $\ln x$.  While this model is known
to require corrections for small primes (the Hardy--Littlewood
singular series), its large-scale predictions have been validated
empirically for primes up to ${\sim}10^{18}$~\cite{Oliveira2014}.

All such validations, however, probe primes of at most 19~digits.
The question of whether the Cram\'er model remains valid at
\emph{extreme} digit counts---hundreds of digits---has been entirely
open, because randomly sampling primes of 500~digits and testing their
neighbors is computationally prohibitive.

The Titan Project provides a natural laboratory for this question.
The polynomial $Q(n) = n^{47} - (n{-}1)^{47}$ generates probable
primes of 480--520 digits at a density that makes systematic
neighborhood scanning feasible.  Each such prime~$P = Q(n)$ serves
as a ``main star,'' and we search its vicinity $[P - R, P]$ for
``satellite primes'' $P - k$ that are also probable primes.

This paper (Part~IV) reports the results of scanning $R = 5000$
around 2{,}107 main-star primes, yielding 9{,}012 satellites.
A complementary confirmation scan at $R = 100$ across the
full 2{,}992 main-star catalog reveals 7~twin prime pairs and
7~sexy prime pairs among 500-digit primes.
The central finding is that \textbf{the local prime landscape at
500-digit scales is indistinguishable from the Cram\'er random
model}---there is no detectable bias, repulsion, or attraction
from the polynomial origin of~$P$.

\subsection{Prior Work}

Parts~I--III of the Titan Project~\cite{Chen2026a, Chen2026b, Chen2026c}
established the constellation hierarchy for $Q(n)$: 742~quadruplets,
7~quintuplets, and 0~sextuplets in $n \leq 2 \times 10^{11}$, all
consistent with the Bateman--Horn conjecture.  The present paper
shifts focus from the \emph{internal} structure of
$Q$-value constellations to the \emph{external} prime landscape
surrounding each $Q$-value prime.


%==========================================================================
\section{Method and Data}\label{sec:method}
%==========================================================================

\subsection{The Titan Radar}

For each main-star prime $P = Q(n)$, the
\texttt{titan\_radar\_ultimate\_5000.py} script tests all candidates
$P - k$ for $k = 2, 4, 6, \ldots, 5000$ (even~$k$ only, since $P$
is odd) using 25-round Miller--Rabin probabilistic primality tests.
The false-positive probability per test is at most $4^{-25} < 10^{-15}$,
giving an expected 0 false positives across the entire survey.

\begin{remark}[Left-side scanning]
The survey tests only $P - k$ (primes below~$P$), not $P + k$.
This is a \emph{statistical}, not structural, limitation.  The
Cram\'er model is symmetric: $P + k$ has an analogous forbidden
lattice ($k \equiv 0$ or $4 \pmod{6}$, the same admissible fraction
$2/3$), and the conditional HL analysis applies identically.
One-sided scanning yields unbiased estimates of all Cram\'er
parameters, with $\sqrt{2}$ larger confidence intervals than a
two-sided scan.  The left side was chosen because the satellite
data were collected as a byproduct of the downward primality
search in Parts~II--III\@.  A right-side scan is proposed as
future work (\S\ref{sec:discussion}).
\end{remark}

\subsection{Survey Parameters}

\begin{table}[H]
\centering
\caption{Survey parameters.}\label{tab:survey}
\smallskip
\begin{tabular}{ll}
\toprule
Main-star primes scanned & 2{,}107 (inferred, see \S\ref{sec:poisson}) \\
Stars yielding $\geq 1$ satellite & 2{,}079 (directly observed) \\
Range of~$n$ & $[5.29 \times 10^{10},\; 2.00 \times 10^{11}]$ \\
Digit range of~$P$ & 494--521 \\
Search radius~$R$ & 5{,}000 \\
Candidates tested per star & 2{,}500 (even $k$) \\
Total satellites found & 9{,}012 \\
Miller--Rabin rounds & 25 \\
\bottomrule
\end{tabular}
\end{table}

\subsection{Confirmation Scan at $R = 100$}

To verify the close-encounter statistics with the full
main-star catalog, a second scan was performed using
\texttt{titan\_radar\_ultimate\_100.py} with $R = 100$
(testing $k = 2, 4, \ldots, 100$).  This scan covered
all 2{,}992 main-star primes ($748 \times 4$ from the complete
quadruplet census, $n$ from $2.19 \times 10^{8}$ to
$2.00 \times 10^{11}$; digit range 376--521),
yielding 235 satellites in 59~seconds.  The close-encounter
counts ($k \leq 100$) from this scan supersede those from
the $R = 5000$ survey, which covered only a partial subset
of 2{,}107 stars.


%==========================================================================
\section{The Forbidden Residue Lattice}\label{sec:lattice}
%==========================================================================

\subsection{Fixed Residues of $Q(n)$}

A necessary condition for $P - k$ to be prime (and $> 5$) is that
$P - k$ must be coprime to $2, 3, 5$.  The fixed residues of~$Q$
modulo small primes constrain the admissible gap values.

\begin{proposition}[Fixed Residues]\label{prop:residues}
For all positive integers~$n$:
\begin{enumerate}[label=(\alph*)]
  \item $Q(n) \equiv 1 \pmod{2}$ \quad (all $Q$-values are odd);
  \item $Q(n) \equiv 1 \pmod{3}$ \quad (all $Q$-values are
    $\equiv 1 \bmod 3$).
\end{enumerate}
\end{proposition}

\begin{proof}
Part~(a): $n^{47}$ and $(n{-}1)^{47}$ have opposite parities,
so their difference is odd.
Part~(b): Since $3 \not\equiv 1 \pmod{47}$, by
Theorem~2.1 of~\cite{Chen2026c}, $Q(n)$ has no root modulo~3.
Checking $Q(0) = 1, Q(1) = 1, Q(2) = 1 \pmod{3}$ shows that
$Q(n) \equiv 1$ for all~$n$.
\end{proof}

These fixed residues impose a \emph{forbidden residue lattice} on
the gap~$k$:
\begin{itemize}[nosep]
  \item $P$ odd $\Rightarrow$ $P - k$ odd iff $k$ even;
  \item $P \equiv 1 \pmod{3}$ $\Rightarrow$ $P - k \not\equiv 0
    \pmod{3}$ iff $k \not\equiv 1 \pmod{3}$.
\end{itemize}
Combining these constraints:
\begin{equation}\label{eq:admissible}
  k \text{ is admissible} \iff k \equiv 0 \text{ or } 2 \pmod{6}.
\end{equation}
Out of the 2{,}500 even integers in $[2, 5000]$, exactly
$\lfloor 5000/6 \rfloor \times 2 + \text{boundary} = 1{,}667$ are
admissible---one-third of even gaps are forbidden.

\begin{remark}
The mod-5 residue of $Q(n)$ varies with~$n$ (taking values
1, 2, or 4), so the mod-5 constraint on~$k$ depends on the
individual star and does not produce a universal forbidden class.
\end{remark}

\subsection{Empirical Verification}

All 9{,}012 satellite gaps satisfy $k \equiv 0$ or $2 \pmod{6}$
with zero exceptions, providing an empirical confirmation of
Proposition~\ref{prop:residues} across all 2{,}107 main stars.
No satellite was found at any $k \equiv 4 \pmod{6}$ (i.e.,
$k = 4, 10, 16, 22, \ldots$), as predicted.


%==========================================================================
\section{Gap Distribution: The Cram\'er Model at 500 Digits}\label{sec:cramer}
%==========================================================================

\subsection{Uniform Density Within the Search Radius}

The Cram\'er model predicts that, conditioned on~$P$ being prime,
the probability that a nearby integer~$m$ is also prime is
approximately $1/\ln P$.  Since $\ln P \approx d \cdot \ln 10
\approx 1{,}150$ for 500-digit primes, the expected number of
primes in an interval of length~$R = 5000$ is
$R/\ln P \approx 4.35$.

The forbidden residue lattice (\S\ref{sec:lattice}) restricts
satellites to 1{,}667 admissible slots out of 2{,}500 even
integers in $[2, 5000]$.  The prime density that would otherwise
be distributed across all integers is \emph{concentrated} onto
these admissible slots: each admissible~$k$ has approximately
$3/\ln P \approx 3/1150 \approx 0.0026$ probability of yielding
a satellite.  This is three times the na\"{\i}ve $1/\ln P$ rate,
precisely because the lattice eliminates two-thirds of the
candidates and redistributes their share.

This predicts a \emph{uniform} distribution of gaps across the
admissible~$k$ within $[2, 5000]$.
Figure~\ref{fig:gaps}(a) confirms this:
a $\chi^2$ goodness-of-fit test on 10 equal-width bins yields
$\chi^2 = 10.48$, $p = 0.31$, showing no significant deviation
from uniformity.

\begin{figure}[H]
\centering
\includegraphics[width=\textwidth]{figures/p3_fig1.png}
\caption{(a)~Histogram of satellite gaps (250-bin width) with
the uniform reference level.  The $\chi^2$ test ($p = 0.31$)
confirms no deviation from uniformity.
(b)~Distribution of gaps modulo~30, showing the 10 admissible
residue classes.  Red bars: $k \equiv 0 \pmod{6}$;
blue bars: $k \equiv 2 \pmod{6}$.}\label{fig:gaps}
\end{figure}

\subsection{Mod-30 Fine Structure}

Figure~\ref{fig:gaps}(b) reveals a non-uniform distribution among
the 10 admissible residue classes modulo~30.
Table~\ref{tab:mod30} reports the counts.

\begin{table}[H]
\centering
\caption{Satellite counts by gap residue modulo~30.  The non-uniformity
reflects the variable mod-5 residue of $Q(n)$.}\label{tab:mod30}
\smallskip
\begin{tabular}{ccrr}
\toprule
$k \bmod 30$ & $k \bmod 6$ & Count & Fraction \\
\midrule
0  & 0 & 1{,}143 & 12.7\% \\
8  & 2 & 1{,}186 & 13.2\% \\
20 & 2 & 1{,}121 & 12.4\% \\
18 & 0 & 1{,}086 & 12.1\% \\
14 & 2 &   892   & 9.9\% \\
24 & 0 &   891   & 9.9\% \\
6  & 0 &   705   & 7.8\% \\
26 & 2 &   698   & 7.7\% \\
2  & 2 &   647   & 7.2\% \\
12 & 0 &   643   & 7.1\% \\
\bottomrule
\end{tabular}
\end{table}

The four most frequent residues ($k \equiv 0, 8, 18, 20 \pmod{30}$)
each exceed 12\%, while the four least frequent
($k \equiv 2, 6, 12, 26$) cluster near 7\%.
This two-tier structure arises from the mod-5 filtering: since
$Q(n) \bmod 5 \in \{1, 2, 4\}$ (never~0 or~3), certain
$k \bmod 5$ classes are suppressed for specific~$n$ values.
When averaged over the survey, the non-uniform mod-5 distribution
of main stars creates the observed asymmetry.


%==========================================================================
\section{Satellite Count Statistics}\label{sec:poisson}
%==========================================================================

\subsection{Expected Satellite Count}

Under the Cram\'er model, the number of primes in an interval
of length~$R$ near~$P$ is approximately Poisson with parameter
\begin{equation}\label{eq:lambda}
  \lambda = \frac{R}{\ln P}
  \approx \frac{5000}{d \cdot \ln 10}
  \approx \frac{5000}{1150}
  \approx 4.35.
\end{equation}

\subsection{Recovery of Zero-Satellite Stars}

The raw satellite log contains 9{,}012 records grouped under
2{,}079 distinct main-star values of~$n$.  A na\"{\i}ve
\texttt{groupby(n).count()} aggregation would report
$N = 2{,}079$---but this is a \emph{data-aggregation artifact}:
stars that produced zero satellites within $R = 5000$ leave no
record in the log and are silently dropped from the count.

Under the Poisson model, the probability of zero satellites is
$e^{-\lambda} \approx 0.013$, so the true total number of
scanned stars is
\[
  N_{\text{total}} = \frac{N_{\text{with}}}{1 - e^{-\lambda}}
  = \frac{2079}{1 - e^{-4.28}} \approx 2{,}107,
\]
implying ${\sim}28$ zero-satellite stars were lost in aggregation.
We adopt $N_{\text{total}} = 2{,}107$ and the corrected
$\lambda = 9012/2107 = 4.278$.

\subsection{Poisson Fit}

Table~\ref{tab:poisson} compares the observed frequency distribution
(with the ${\sim}28$ zero-satellite stars restored) against the
Poisson prediction.

\begin{table}[H]
\centering
\caption{Satellite count distribution: observed vs.\ Poisson
($\lambda = 4.28$, $N = 2{,}107$).  The $k = 0$ bin is
reconstructed from the aggregation correction.}\label{tab:poisson}
\smallskip
\begin{tabular}{crrc}
\toprule
Satellites & Observed & Poisson & Ratio \\
\midrule
0  &  28$^*$ &  29  & 0.96 \\
1  & 135     & 125  & 1.08 \\
2  & 283     & 267  & 1.06 \\
3  & 367     & 381  & 0.96 \\
4  & 406     & 408  & 1.00 \\
5  & 341     & 349  & 0.98 \\
6  & 251     & 249  & 1.01 \\
7  & 131     & 152  & 0.86 \\
8  &  84     &  81  & 1.03 \\
9  &  51     &  39  & 1.32 \\
10 &  18     &  17  & 1.09 \\
$\geq$11 & 12 & 10 & 1.19 \\
\bottomrule
\multicolumn{4}{l}{\footnotesize $^*$Inferred from $N_{\text{total}} - N_{\text{with}} = 2107 - 2079 = 28$.}
\end{tabular}
\end{table}

The dispersion index (variance-to-mean ratio) over the full
$N = 2{,}107$ sample is
$\sigma^2/\mu = 4.57/4.28 = 1.07$, compared to the Poisson
value of~1.  The fit is excellent across all 12~bins,
with no bin deviating by more than a factor of 1.3.
The recovery of the zero-satellite bin eliminates the sole
anomaly in the original fit and produces a textbook-quality
Poisson distribution at 500-digit scales.

\begin{figure}[H]
\centering
\includegraphics[width=\textwidth]{figures/p3_fig2.png}
\caption{(a)~CDF of the nearest-satellite gap, compared with the
Cram\'er exponential $1 - \exp(-k/3\ln P)$.
(b)~Satellite-count histogram (blue) vs.\ Poisson prediction
(red), $\lambda = 4.28$.  Dispersion index~=~1.07.}\label{fig:poisson}
\end{figure}


%==========================================================================
\section{Satellite Density vs.\ Main-Star Size}\label{sec:density}
%==========================================================================

The Cram\'er model predicts that the satellite density decreases
as $\ln P$ grows with~$n$.  Table~\ref{tab:density} confirms this.

\begin{table}[H]
\centering
\caption{Mean satellite count by $n$-range, compared with the
Cram\'er prediction $R/\ln P$.}\label{tab:density}
\smallskip
\begin{tabular}{lcccr}
\toprule
$n$-range (B) & Stars & Mean digits & Observed & Cram\'er \\
\midrule
$[50, 75)$   & 290 & 498 & 4.62 & 4.36 \\
$[75, 100)$  & 403 & 505 & 4.35 & 4.31 \\
$[100, 125)$ & 344 & 512 & 4.34 & 4.26 \\
$[125, 150)$ & 391 & 514 & 4.34 & 4.23 \\
$[150, 175)$ & 322 & 517 & 4.21 & 4.20 \\
$[175, 200)$ & 329 & 520 & 4.17 & 4.18 \\
\bottomrule
\end{tabular}
\end{table}

The observed-to-Cram\'er ratio decreases monotonically from 1.06
to 1.00 across the range, with a survey-wide mean of 1.019.
Figure~\ref{fig:density} visualizes this remarkable agreement.

\begin{figure}[H]
\centering
\includegraphics[width=\textwidth]{figures/p3_fig3.png}
\caption{(a)~Satellite count per star (dots) with binned means
(red) and the Cram\'er prediction $R/\ln P$ (green dashed).
(b)~Observed/Cram\'er ratio by $n$-region; all ratios lie within
$[0.95, 1.10]$.}\label{fig:density}
\end{figure}




%==========================================================================
\section{Close Encounters: The Twin--Sexy Symmetry}%
\label{sec:close}
%==========================================================================

\subsection{The Smallest Gaps}

The closest satellite encounters probe the extreme tail of the
prime gap distribution at 500-digit scales.
Table~\ref{tab:close} lists all satellites with $k \leq 30$
from the $R = 100$ confirmation scan (2{,}992 stars).

\begin{table}[H]
\centering
\caption{Close satellite encounters ($k \leq 30$) from the
$R = 100$ scan across 2{,}992 main stars.
All gaps are admissible ($k \equiv 0$ or $2 \bmod 6$).
}\label{tab:close}
\smallskip
\begin{tabular}{cccc}
\toprule
$k$ & Name & Count & Example $n$ \\
\midrule
2  & Twin    & \textbf{7} & 41{,}262{,}186{,}068 \\
6  & Sexy    & \textbf{7} & 103{,}957{,}400{,}503 \\
8  & Octet   & 2 &  93{,}134{,}573{,}699 \\
12 & ---     & 6 & 111{,}807{,}642{,}623 \\
14 & ---     & 13 &  55{,}109{,}556{,}141 \\
18 & ---     & 7 &  66{,}483{,}948{,}763 \\
20 & ---     & 11 &  53{,}309{,}481{,}336 \\
24 & ---     & 3 & 105{,}463{,}974{,}582 \\
26 & ---     & 12 &  46{,}880{,}923{,}697 \\
30 & ---     & 7 &  13{,}217{,}014{,}958 \\
\bottomrule
\end{tabular}
\end{table}

\subsection{Conditional Hardy--Littlewood: The Bayesian Doubling Effect}

Before analyzing specific gaps, we establish a key principle
governing satellite statistics in our fixed-residue setting.

\begin{proposition}[Conditional Concentration]\label{prop:bayes}
For the admissible gap classes:
\begin{enumerate}[label=(\alph*)]
  \item If $k \equiv 0 \pmod{3}$ (i.e., $k \equiv 0 \pmod{6}$):
    both $P \equiv 1$ and $P \equiv 2 \pmod{3}$ allow $P - k$
    coprime to~3.  No concentration occurs; the Bayesian
    factor is~$B = 1$.
  \item If $k \equiv 2 \pmod{3}$ (i.e., $k \equiv 2 \pmod{6}$):
    only $P \equiv 1 \pmod{3}$ allows $P - k$ coprime to~3
    (since $P - k \equiv P - 2 \equiv -1 \pmod{3}$, which is
    nonzero; whereas $P \equiv 2$ gives $P - k \equiv 0$).
    All pairs $(P, P{-}k)$ with both prime must have
    $P \equiv 1 \pmod{3}$---the Bayesian factor is~$B = \mathbf{2}$.
\end{enumerate}
\end{proposition}

Since every main star satisfies $P \equiv 1 \pmod{3}$
(Proposition~\ref{prop:residues}), the conditional expected count
for gap~$k$ across $N = 2{,}992$ stars is:
\begin{equation}\label{eq:conditional}
  E[k] = N \cdot \frac{\mathcal{S}_{\text{cond}}(k)}{\overline{\ln P}},
  \qquad
  \mathcal{S}_{\text{cond}}(k) = B(k) \cdot
    \prod_{\substack{p \geq 5 \\ p \text{ prime}}}
    \frac{1 - \nu_k(p)/p}{(1 - 1/p)^2},
\end{equation}
where $B(k)$ is the Bayesian factor (2 if $k \equiv 2 \bmod 3$,
1 if $k \equiv 0 \bmod 3$), $\nu_k(p)$ is the number of
residues in $\{0, k\} \bmod p$, and
$\overline{\ln P} \approx 1{,}100$ is the average over the
2{,}992-star catalog.

Table~\ref{tab:conditional} reports the conditional singular
series for the smallest admissible gaps, compared with the
$R = 100$ observations.

\begin{table}[H]
\centering
\caption{Conditional Hardy--Littlewood analysis for small gaps
($N = 2{,}992$ stars, $\overline{\ln P} \approx 1{,}100$).
$\mathcal{S}_{\text{cond}}$ incorporates the Bayesian factor
$B(k)$.}\label{tab:conditional}
\smallskip
\begin{tabular}{cccccrc}
\toprule
$k$ & $k \bmod 6$ & $B(k)$ &
$\mathcal{S}_{\text{cond}}$ & $E[k]$ & Obs. & $\sigma$ \\
\midrule
2  & 2 & 2 & 2.64 & 7.2 &  \textbf{7} & $-0.1$ \\
6  & 0 & 1 & 2.64 & 7.2 &  \textbf{7} & $-0.1$ \\
8  & 2 & 2 & 2.64 & 7.2 &  2 & $-1.9$ \\
12 & 0 & 1 & 2.64 & 7.2 &  6 & $-0.4$ \\
14 & 2 & 2 & 3.17 & 8.6 & 13 & $+1.5$ \\
18 & 0 & 1 & 2.64 & 7.2 &  \textbf{7} & $-0.1$ \\
20 & 2 & 2 & 3.52 & 9.6 & 11 & $+0.5$ \\
24 & 0 & 1 & 2.64 & 7.2 &  3 & $-1.6$ \\
26 & 2 & 2 & 2.88 & 7.8 & 12 & $+1.5$ \\
30 & 0 & 1 & 3.52 & 9.6 &  7 & $-0.8$ \\
\bottomrule
\end{tabular}
\end{table}

\begin{remark}[The mod-6 cancellation]
The unconditional HL singular series $\mathcal{S}(k)$ is
systematically larger for $k \equiv 0 \pmod{6}$ than for
$k \equiv 2 \pmod{6}$ (by a factor of ${\sim}2$), because the
former avoids the $p = 3$ sieve penalty.  But the Bayesian
doubling for $k \equiv 2 \pmod{6}$ exactly compensates this
deficit: $\mathcal{S}_{\text{cond}}(k{=}2) = 2 \times 1.32 = 2.64
= \mathcal{S}_{\text{cond}}(k{=}6) = 1 \times 2.64$.  This
cancellation explains the empirically observed near-equality
of satellite counts between the two mod-6 classes
(4{,}468 vs.\ 4{,}544; ratio 1.02) in the $R = 5000$ survey.
\end{remark}

\subsection{Seven Twin Prime Pairs}\label{sec:twin}

Seven twin prime pairs ($k = 2$) were identified across
the 2{,}992-star catalog.
Table~\ref{tab:twins} lists the complete census.

\begin{table}[H]
\centering
\caption{Complete twin prime satellite catalog ($k = 2$):
all 7 pairs $(P,\, P-2)$ with both members
of ${\sim}500$~digits.}\label{tab:twins}
\smallskip
\begin{tabular}{clc}
\toprule
\# & Main-star $n$ & Approx.\ digits \\
\midrule
1 &  41{,}262{,}186{,}068 & 498 \\
2 &  63{,}150{,}957{,}871 & 507 \\
3 &  68{,}875{,}255{,}098 & 509 \\
4 & 123{,}037{,}305{,}946 & 521 \\
5 & 124{,}340{,}002{,}320 & 521 \\
6 & 126{,}720{,}185{,}653 & 521 \\
7 & 193{,}087{,}289{,}846 & 530 \\
\bottomrule
\end{tabular}
\end{table}

The conditional expectation is $E[k{=}2] = 7.2$
(Table~\ref{tab:conditional}); the observed 7 lies within
$0.1\sigma$---an essentially perfect match.  These are among
the largest twin prime pairs found by systematic census
rather than targeted record searches.

\subsection{Seven Sexy Prime Pairs}\label{sec:sexy}

Seven sexy prime pairs ($k = 6$) were identified.
Table~\ref{tab:sexys} lists the complete census.

\begin{table}[H]
\centering
\caption{Complete sexy prime satellite catalog ($k = 6$):
all 7 pairs $(P,\, P-6)$ with both members
of ${\sim}500$~digits.}\label{tab:sexys}
\smallskip
\begin{tabular}{clc}
\toprule
\# & Main-star $n$ & Approx.\ digits \\
\midrule
1 &  29{,}707{,}259{,}863 & 492 \\
2 & 103{,}957{,}400{,}503 & 518 \\
3 & 105{,}463{,}974{,}584 & 518 \\
4 & 122{,}726{,}858{,}404 & 521 \\
5 & 152{,}789{,}753{,}532 & 524 \\
6 & 154{,}849{,}622{,}427 & 525 \\
7 & 166{,}607{,}083{,}748 & 526 \\
\bottomrule
\end{tabular}
\end{table}

The conditional expectation is $E[k{=}6] = 7.2$, identical
to $E[k{=}2]$ by the mod-6 cancellation; the observed 7 again
matches to $0.1\sigma$.

\subsection{The Twin--Sexy Symmetry: $N_{\text{twin}} = N_{\text{sexy}} = 7$}

The equality $N_{\text{twin}} = N_{\text{sexy}} = 7$ is the
headline result of the close-encounter analysis.  It is a direct
empirical confirmation of the identity
$\mathcal{S}_{\text{cond}}(k{=}2) \equiv
\mathcal{S}_{\text{cond}}(k{=}6)$, which the theory predicts
must hold exactly.

To quantify: under independent Poisson draws with
$\lambda = 7.2$, the probability of observing the \emph{same}
count for both $k = 2$ and $k = 6$ is $\sum_j P(j)^2 \approx 11\%$.
The specific value $j = 7$ has single-event probability
$P(7) = 14.9\%$, making the joint observation a
$14.9\%^2 = 2.2\%$ event---unusual but not extraordinary.
Notably, the quintuplet count from Part~III~\cite{Chen2026c}
is also~7, giving a triple coincidence $7 = 7 = 7$, though
quintuplets arise from an independent mechanism and the numerical
agreement is coincidental.

\subsection{The 3-Smooth Baseline Family and the $k = 8$ Deficit}%
\label{sec:smooth}

The gaps $k = 2, 6, 8, 12, 18, 24, 32, 36, 48, 54, 72, 96$ share
a common property: their only prime factors are 2 and~3 (they are
\emph{3-smooth}).  For all such~$k$:
\[
  \mathcal{S}_{\text{cond}}(k) = 2.64
  \qquad\text{(identical baseline)},
\]
because no prime $p \geq 5$ divides~$k$, so the HL product over
$p \geq 5$ contributes the same factor for each, and the $p = 3$
effect is absorbed by the Bayesian factor $B(k)$.

Table~\ref{tab:smooth} shows the 12 members of this family
within $k \leq 100$.

\begin{table}[H]
\centering
\caption{The 3-smooth baseline family ($k = 2^a \cdot 3^b$,
$a \geq 1$; all have $\mathcal{S}_{\text{cond}} = 2.64$,
$E \approx 7.2$).}\label{tab:smooth}
\smallskip
\begin{tabular}{ccccrc}
\toprule
$k$ & Factorization & $B(k)$ & $E$ & Obs. & $\sigma$ \\
\midrule
2  & $2$             & 2 & 7.2 & 7 & $-0.1$ \\
6  & $2 \times 3$    & 1 & 7.2 & 7 & $-0.1$ \\
8  & $2^3$           & 2 & 7.2 & 2 & $-1.9$ \\
12 & $2^2 \times 3$  & 1 & 7.2 & 6 & $-0.4$ \\
18 & $2 \times 3^2$  & 1 & 7.2 & 7 & $-0.1$ \\
24 & $2^3 \times 3$  & 1 & 7.2 & 3 & $-1.6$ \\
32 & $2^5$           & 2 & 7.2 & 5 & $-0.8$ \\
36 & $2^2 \times 3^2$& 1 & 7.2 & 6 & $-0.4$ \\
48 & $2^4 \times 3$  & 1 & 7.2 & 0 & $-2.7$ \\
54 & $2 \times 3^3$  & 1 & 7.2 & 8 & $+0.3$ \\
72 & $2^3 \times 3^2$& 1 & 7.2 & 8 & $+0.3$ \\
96 & $2^5 \times 3$  & 1 & 7.2 & 11& $+1.4$ \\
\midrule
\multicolumn{3}{c}{$\Sigma$} & 86.2 & 70 & $-1.7$ \\
\bottomrule
\end{tabular}
\end{table}

Three members ($k = 2, 6, 18$) attain exactly the mode
$\text{obs} = 7 \approx E$; the overall $\chi^2 = 16.6$
on 11~d.f.\ ($\chi^2/\text{d.f.} = 1.51$) is consistent
with Poisson scatter.  The $k = 8$ deficit (2 observed,
$-1.9\sigma$) and the $k = 48$ absence (0 observed,
$-2.7\sigma$) are the two largest fluctuations, but among
12~independent Poisson draws one expects ${\sim}1$ to exceed
$1.5\sigma$; the observed pattern is normal.

\begin{remark}[Why $k = 8$ is rarer than $k = 2$]
There is no theoretical mechanism suppressing $k = 8$ relative
to $k = 2$: both have identical $\mathcal{S}_{\text{cond}} = 2.64$
and identical conditional expectation.  The deficit is pure
Poisson fluctuation.  Similarly, $k = 48$ (0~observed)
is the most extreme draw in the baseline family, but
$P(X = 0 \mid \lambda = 7.2) = 0.07\%$, and the probability
that at least one of 12~draws achieves this is ${\sim}4\%$---small
but not anomalous.
\end{remark}


%==========================================================================
\section{The Nearest-Satellite Distribution}\label{sec:nearest}
%==========================================================================

For each main star, the \emph{nearest satellite} gap
$k_{\min} = \min\{k : P - k \text{ is prime}\}$ probes the first
prime below~$P$.  Under the Cram\'er model, the CDF of $k_{\min}$
over many stars follows
\begin{equation}\label{eq:cdf}
  F(k) = P(k_{\min} \leq k)
  = 1 - \exp\!\left(-\frac{k}{f_{\text{adm}} \cdot \ln P}\right),
\end{equation}
where $f_{\text{adm}} = 30/10 = 3$ is the inverse admissible
fraction (only $10/30$ of even gaps modulo~30 are admissible).

Table~\ref{tab:nearest} compares the observed CDF with this
prediction.

\begin{table}[H]
\centering
\caption{Nearest-satellite CDF: observed vs.\ Cram\'er.
$\ln P \approx 1{,}175$.}\label{tab:nearest}
\smallskip
\begin{tabular}{rccc}
\toprule
$k_{\min} \leq$ & Cram\'er CDF & Observed CDF & Ratio \\
\midrule
50   & 0.042 & 0.033 & 0.80 \\
100  & 0.082 & 0.074 & 0.91 \\
200  & 0.156 & 0.154 & 0.99 \\
500  & 0.347 & 0.357 & 1.03 \\
1000 & 0.573 & 0.582 & 1.01 \\
2000 & 0.818 & 0.825 & 1.01 \\
3000 & 0.925 & 0.931 & 1.01 \\
\bottomrule
\end{tabular}
\end{table}

The agreement is within 1--3\% for $k \geq 100$, with a mild
deficit at $k < 100$ (ratio 0.80--0.91) that may reflect
unmodeled Hardy--Littlewood corrections at the smallest gaps.
Figure~\ref{fig:poisson}(a) overlays the observed CDF on the
theoretical curve.


%==========================================================================
\section{Discussion}\label{sec:discussion}
%==========================================================================

\subsection{Cram\'er Universality at Extreme Scales}

The principal result of this study is the \emph{Cram\'er
universality} of the local prime landscape at 500-digit scales:
every statistical signature tested---gap uniformity, Poisson
count distribution, nearest-neighbor CDF, density variation
with~$n$---agrees with the Cram\'er random model to within a few
percent.

This is significant because the main-star primes are not ``random''
numbers: they are values of a specific degree-46 polynomial,
with structured residue classes (e.g., $P \equiv 1 \bmod{6}$)
and a bifurcated root structure at resonant
primes~\cite{Chen2026c}.  Yet none of this algebraic structure
is detectable in the satellite statistics---neither in the bulk gap
distribution nor in the close-encounter regime where the
twin--sexy symmetry provides a precision test at $0.1\sigma$.
The local prime environment treats~$P$ as if it were a generic
integer of the same magnitude.

A crucial distinction underlies this universality: the satellite
primes $P - k$ are \emph{not} values of the polynomial~$Q$.
They are ordinary integers whose primality depends only on
their size (${\sim}10^{500}$) and their residue classes modulo
small primes---not on the algebraic origin of~$P$.  The
Bateman--Horn polynomial constant $C(Q)$, which governs the
density of $Q$-value primes (Parts~I--III), plays no role in
satellite statistics.  The conditional HL
analysis~\eqref{eq:conditional} requires only two inputs from
the polynomial: the magnitude of~$P$ (setting $\ln P$) and the
fixed residue $P \equiv 1 \pmod{6}$ (determining the Bayesian
factor and the forbidden lattice).  All other algebraic
structure is provably invisible to the satellite census.

\subsection{The Bayesian Concentration Principle}

The conditional HL analysis (Proposition~\ref{prop:bayes}) reveals
a subtle but beautiful phenomenon: the fixed residue
$P \equiv 1 \pmod{3}$ creates a \emph{Bayesian doubling} for all
gaps $k \equiv 2 \pmod{6}$ (including twin primes), because
100\% of such pairs among all primes concentrate in the
$P \equiv 1 \pmod{3}$ subspace.  This doubling is exactly
compensated by the larger unconditional HL factor for
$k \equiv 0 \pmod{6}$ gaps, producing the observed near-equality
of the two mod-6 classes (4{,}468 vs.\ 4{,}544).

This cancellation is not a coincidence but a structural identity:
for any fixed residue $P \equiv a \pmod{3}$, the conditional
singular series $\mathcal{S}_{\text{cond}}(k)$ depends on~$k$ only
through the primes $p \geq 5$, making the mod-3 effects invisible
in the aggregate gap distribution---precisely as observed.

\subsection{The Forbidden Lattice as a Natural Sieve}

The constraint $k \equiv 0$ or $2 \pmod{6}$ reduces the
effective search window from 2{,}500 candidates to 1{,}667
admissible ones---a 33\% reduction.  Crucially, this does not
reduce the satellite density: the forbidden lattice
\emph{concentrates} the same total prime density onto fewer
admissible slots, boosting each slot's hit rate to
${\sim}3/\ln P$ (three times the na\"{\i}ve $1/\ln P$).

From a theoretical perspective, this forbidden lattice is a
one-dimensional analog of the admissibility conditions in the
Hardy--Littlewood prime $k$-tuples
conjecture~\cite{HardyLittlewood1923}: just as a $k$-tuple must
avoid covering all residues modulo any prime, the satellite gaps
must avoid the forbidden residues imposed by the main star's
fixed congruences.

\subsection{The Twin--Sexy Symmetry as a Precision Test}

The equality $N_{\text{twin}} = N_{\text{sexy}} = 7$ against
$E = 7.2$ constitutes a $0.1\sigma$ match---the most precise
validation of the conditional Hardy--Littlewood framework in
this study.  The identity
$\mathcal{S}_{\text{cond}}(k{=}2) = \mathcal{S}_{\text{cond}}(k{=}6)$
follows from a cancellation between the Bayesian doubling ($B = 2$
for twin primes) and the larger unconditional singular series
($\mathcal{S}(k{=}6) = 2\mathcal{S}(k{=}2)$ from the $p = 3$
factor).  The data confirm this identity to remarkable accuracy.

More broadly, all 12 members of the 3-smooth baseline family
($\S$\ref{sec:smooth}) share the same conditional expectation
$E = 7.2$, and their observed distribution is consistent with
Poisson scatter ($\chi^2/\text{d.f.} = 1.51$).  The two
outliers---$k = 8$ (2~observed) and $k = 48$ (0~observed)---are
within the expected range of fluctuations for 12 independent draws.

\subsection{Computational Remarks}

Each main star requires testing 2{,}500 candidates at
${\sim}15$\,ms per 25-round Miller--Rabin test (500-digit numbers),
for a total of ${\sim}37.5$ seconds per star.  The full survey of
2{,}107 stars consumed approximately 22~CPU-hours on a single
core.  The data were collected as a byproduct of the deep-space
quadruplet search~\cite{Chen2026b}, with negligible additional
cost.

A natural extension is the \emph{right-side scan} ($P + k$),
which would test the symmetry of the Cram\'er model around~$P$
and double the statistical power for rare-gap detection.

\begin{figure}[H]
\centering
\includegraphics[width=\textwidth]{figures/p3_fig4.png}
\caption{(a)~All close encounters ($k \leq 100$), with $k = 2$
(twin primes) and $k = 6$ (sexy primes) marked.
(b)~Fine-grained gap census for $k < 62$.  Seven twin pairs
($k = 2$) and seven sexy pairs ($k = 6$) constitute the closest
encounters, confirming
$\mathcal{S}_{\text{cond}}(2) = \mathcal{S}_{\text{cond}}(6)$.}\label{fig:small}
\end{figure}


%==========================================================================
\section{Conclusion}
%==========================================================================

The satellite prime survey---9{,}012 primes within radius~5{,}000
of 2{,}107 giant primes generated by $Q(n) = n^{47} - (n{-}1)^{47}$,
supplemented by a close-encounter scan of all 2{,}992 main stars
at radius~100---provides the first large-scale validation of the
Cram\'er random model at 500-digit scales.

The key findings are:
\begin{enumerate}[nosep]
  \item \textbf{Cram\'er agreement:} satellite count, gap distribution,
    nearest-neighbor CDF, and density scaling with~$n$ all match the
    Cram\'er model to ${\sim}2\%$.
  \item \textbf{Poisson fit:} the satellite count per star is
    Poisson with $\lambda = 4.28$ and dispersion index 1.07, including
    ${\sim}28$ zero-satellite stars recovered from a data-aggregation
    artifact.
  \item \textbf{Forbidden residue lattice:} all gaps satisfy
    $k \equiv 0$ or $2 \pmod{6}$, a consequence of
    $Q(n) \equiv 1 \pmod{6}$, concentrating satellite density onto
    $1{,}667/2{,}500$ admissible slots at rate ${\sim}3/\ln P$ each.
  \item \textbf{Bayesian concentration:} the fixed residue
    $P \equiv 1 \pmod{3}$ doubles the conditional twin-prime rate
    for $k \equiv 2 \pmod{6}$ gaps, exactly compensating the
    smaller unconditional HL factor---explaining the observed
    equality of the two mod-6 classes.
  \item \textbf{Twin--sexy symmetry:} 7~twin pairs ($k = 2$) and
    7~sexy pairs ($k = 6$) of 500-digit primes were found, both
    matching the conditional expectation $E = 7.2$ to within
    $0.1\sigma$---a precision confirmation of the identity
    $\mathcal{S}_{\text{cond}}(k{=}2) = \mathcal{S}_{\text{cond}}(k{=}6)$.
  \item \textbf{3-smooth baseline:}  all 12~gaps in the 3-smooth
    family ($k = 2^a \cdot 3^b$) share the same
    $\mathcal{S}_{\text{cond}} = 2.64$; the observed scatter
    ($\chi^2/\text{d.f.} = 1.51$) is consistent with Poisson
    fluctuation, explaining why $k = 8$ (2~obs.) is rarer than
    $k = 2$ (7~obs.) despite identical expectations.
\end{enumerate}

These results establish that the algebraic origin of giant primes
from a high-degree polynomial does not perturb their local prime
environment: Cram\'er universality holds at 500~digits.


%==========================================================================
\section*{Data and Code Availability}
%==========================================================================

Complete satellite data, analysis scripts, and figures:
\begin{center}
\small
\url{https://github.com/Ruqing1963/Q47-Satellite-Primes}
\end{center}
\noindent Related repositories:
\begin{center}
\small
Part~II: \url{https://github.com/Ruqing1963/Q47-Deep-Space-Quadruplet-Census} \\
Part~III: \url{https://github.com/Ruqing1963/Q47-Quintuplet-Sextuplet-Boundary}
\end{center}
\noindent Zenodo archives:
Part~III: \url{https://zenodo.org/records/18728917},
Part~II: \url{https://zenodo.org/records/18728540},
Part~I: \url{https://zenodo.org/records/18701355}.


%==========================================================================
\begin{thebibliography}{99}
%==========================================================================

\bibitem{Cramer1936}
H. Cram\'er,
``On the order of magnitude of the difference between consecutive prime numbers,''
\textit{Acta Arithmetica}, vol.~2, pp.~23--46, 1936.

\bibitem{HardyLittlewood1923}
G.\,H. Hardy and J.\,E. Littlewood,
``Some problems of `Partitio Numerorum'; III,''
\textit{Acta Math.}, vol.~44, pp.~1--70, 1923.

\bibitem{Oliveira2014}
T. Oliveira e Silva, S. Herzog, and S. Pardi,
``Empirical verification of the even Goldbach conjecture and computation
of prime gaps up to $4 \times 10^{18}$,''
\textit{Math.\ Comp.}, vol.~83, pp.~2033--2060, 2014.

\bibitem{Chen2026a}
R.\,Chen,
``Statistical Morphology and Geodesic Rigidity of Prime Constellations
in $Q(n) = n^{47} - (n-1)^{47}$,''
Zenodo, 2026. \url{https://zenodo.org/records/18701355}.

\bibitem{Chen2026b}
R.\,Chen,
``A Complete Census of Prime Quadruplets for
$Q(n) = n^{47} - (n-1)^{47}$ in $1 \leq n \leq 2 \times 10^{11}$,''
Zenodo, 2026. \url{https://zenodo.org/records/18728540}.

\bibitem{Chen2026c}
R.\,Chen,
``Prime Quintuplets and the Sextuplet Boundary for
$Q(n) = n^{47} - (n-1)^{47}$,''
Zenodo, 2026. \url{https://zenodo.org/records/18728917}.

\bibitem{Maynard2015}
J. Maynard,
``Small gaps between primes,''
\textit{Ann.\ of Math.}, vol.~181, no.~1, pp.~383--413, 2015.

\end{thebibliography}

\end{document}
